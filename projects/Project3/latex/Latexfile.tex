\documentclass[10pt]{article}

\usepackage{amsmath}
\usepackage{graphicx}
\usepackage[margin=0.7in]{geometry}
\usepackage{float}
\usepackage{listings}
\usepackage[utf8]{inputenc}
\usepackage[parfill]{parskip}  
\usepackage{multicol}
\usepackage{siunitx}
\usepackage[dvipsnames]{xcolor}
\usepackage{cleveref}
\usepackage{cite}
\usepackage{caption}
\usepackage{hyperref}
\usepackage{tabularx}
\usepackage{mathabx}

\captionsetup{width=0.6\linewidth}

\newcommand{\rhomax}{\rho_{\text{max}}}
\newcommand{\relerr}{\epsilon_{\text{rel}}}
\newcommand{\bigO}[1]{\mathcal{O}(#1)}

\graphicspath{{../results/}}

\newcommand\myshade{50}
\colorlet{mylinkcolor}{violet}
\colorlet{mycitecolor}{YellowOrange}
\colorlet{myurlcolor}{Aquamarine}

\hypersetup{
  linkcolor  = mylinkcolor!\myshade!black,
  citecolor  = mycitecolor!\myshade!black,
  urlcolor   = myurlcolor!\myshade!black,
  colorlinks = true,
}

\begin{document}
\title{The Solar System
\\ Project 3
\\ FYS3150}
\author{Ragnar Bruvoll \and Halvard Sutterud}
%\author{Halvard Sutterud}
\date{October 2017}
\maketitle{\begin{center}\end{center}}
\pagenumbering{gobble}
\thispagestyle{empty}

\begin{abstract}
    We study a model of the solar system as gravitationally interacting
    celestial bodies.  Using an object oriented approach, we implement
    classes in \texttt{c++} to solve the differential equations of the
    mechanics of the solar system in \texttt{c++}. We compare the effectiveness of
    two methods for numerical differentiation, namely Forward Euler and
    Velocity Verlet, and discuss their stabilities in terms of energy
    conservation. Some properties of the solar system are studied, including
    escape velocity for a planet in a two body system, a modified three
    body problem, and finally the phase precession of a Sun-Mercury system
    with relativistic corrections. 
    The Forward Euler integration method is found to be less costly per
    iteration, but unlike the verlet method it doesn't conserve energy. 
     
    % cases of one and two electrons trapped in a harmonic oscillator.
    % Jacobi's method for finding the eigenvalues of a matrix is
    % implemented, and applied to a discrete approximation of the
    % Schr\"{o}dinger equation. The analytic energies of a single electron
    % is reproduced within a relative error of $\SI{e-4}{}$. Then the
    % method is applied to two interacting electrons in the same potential,
    % where analytic results from \cite{PhysRevA.48.3561} for certain
    % frequencies is reproduced.
\end{abstract}





\begin{multicols}{2}
\tableofcontents

% \newpage
\pagenumbering{arabic}

\section{Introduction}
The purpose of this project is to solve the differential equations of the
mechanics of the solar system.

%%%%%%%%%%%%%%%%%%%%%%%%%%%%%%%%%%%%%%%%%%%%%%%%%%%%%%%%%%%%%%%%%%%%%%%%%%%
\section{Theory/Methods}
\subsection{Units}
 % \subsection{Discretizing equations and defining acceleration}
Using Newton's second law, which states that the sum of all the forces
equals mass times acceleration, $\sum F=ma$, we can calculate the
acceleration of the planet in relation to the Sun. Assuming circular
orbits, we know this acceleration to be the sentripetal acceleration, and
the force is the gravitational force

\begin{equation}
    F_G = M_\Earth a = M_\Earth\frac{v^2}{r} = G\frac{M_\Sun M_\Earth}{r^2}
\end{equation}



with the earth as an example. We will be using the astronomical symbols for
the celestial bodies; $M_\Earth$ is then the mass of the Earth and $M_\Sun$ the mass
of the sun. With this in mind, we can simplify the equations and
calculations by using astronomical units and years as unit lengths. So
instead of using $r = 1.5\cdot10^{11}m$ and , we'll simply assert $r =
1AU$. With simple multiplication, the expressions for sentripetal force and
gravitational force can be written as 

\begin{equation}\label{eq:circVel}
    \Rightarrow v^2r = GM_\Sun
\end{equation}

and with the new unit lengths applied we can simplify it further. We know
that the radius from the earth to it's spin system's centre is one
astronomical unit, 1 AU. The orbit speed is one round per year, that is
2$\pi$AU/year. This means

\begin{equation}
    GM_\Sun = v^2r = 4\pi^2AU^3/year^2
\end{equation}

We can now substitute this constant and get an expression for the general
acceleration of any object with an circular orbit around the Sun

\begin{equation}
    M_\Earth a = GM_\Sun\frac{M_\Earth}{r^2}
\end{equation}

\begin{equation}
    a = \frac{4\pi^2}{r^2}
\end{equation}
with the acceleration in units $[a] = \SI{}{AU/yr^2}$. 

For stability analysis, we will use that the energy of these systems are given by 
\begin{align}\label{eq:energies}
    T &= \sum_{i=1}^N \frac{1}{2}mv_i^2 & V &= \sum_{i=1}^N \sum_{j=i+1}^N
    -r_{i,j}|F_{i,j}|
\end{align}

where $T$ is the total kinetic energy, found by summing over the kinetic
energies of all $N$ objects, and $V$ is the potential energies due to
gravitational force between bodies $F_{i,j}$, given by

\begin{equation}
    F_{i,j} =  G\frac{M_iM_j}{r_{i,j}^2}
\end{equation}

with $M_i$ being the mass of object $i$ and $r_{i,j} = |\vec r_i - \vec
r_j|$ as the distance between two objects.


\subsection{Integration}
As the trajectories of all the objects in the solar system is being
affected by all the other objects at all times, it is quite obvious that
the acceleration is not constant as they would be in a circular orbit. We
therefore need to integrate their position from Newton's equations for each
time step. This can be done with different methods, in our case Euler's
forward algorithm and velocity Verlet method. They're both derived from the
Taylor series. As we cannot make the time steps of the integration
infinitesimaly small, the methods are bound to be incorrect to a certain
degree. The impact of this error will be scrutinized in a later section.

\subsubsection{Taylor expantion}
The Taylor series is an infinite sum of terms given by the function's
derivative at each point. The series is used to express an unknown
curve and it converges closer to the correct value with each degree. 

\begin{equation}
   f(x) =  \sum_{n=0}^\infty\frac{f^{(n)}(a)}{n!}(x-a)^n
\end{equation}

\begin{equation}
    f(x) = f(a)+ \frac{f'(a)}{1!}(x-a)^1+\frac{f''(a)}{2!}(x-a)^2+\cdots
\end{equation}

It is however impossible to add an infinite number of terms, and each term
will further burden the calculating system. But as the value of the terms
decrease for each degree, we can justifiably simplify the series to one of
three terms. This will entail a certain degree of error, but as we will see
it varies between methods, and is often so small we can ignore it.

\subsubsection{Euler's forward algorithm}
The Euler forward method is a first order method, meaning that the local
error is proportional to the step size squared. This incentivizes ut to
pick a step size as small as possible without making the program too slow.
For this method we only use two of the taylor terms, knowing velocity
derived to be acceleration.  

\begin{equation}
    v_{i+1} = v_i +ah
\end{equation}

As for the position, we use the current velocity in the second term to
increase the accuracy.

\begin{equation}
    x_{i+1} = v_i +v_{i+1}h
\end{equation}

With this algorithm we can calculate the position of the object at as many
point as we like, with an increased error with bigger time steps. The
algorithm requires an initial velocity, position and the acceleration at
each point of the position, which we will derive from the forces action
upon on the object. The Euler method has four FLOPS per time step.\\

This method is good for finding an approximated trajectory, but as we will
see, it is far from perfect.


\subsubsection{Velocity Verlet method}
A more accurate method for finding the trajectory, the velocity and
position that is, would be the velocity Verlet method. This too is based on
the Taylor series, with every term from the fourth and on assumed to be
negligible.

\begin{equation}
    v_{i+1} = v_i +hv_i^{(1)}+\frac{h^2}{2}v_i^{(2)}+O(h^3)
\end{equation}

\begin{equation}
    x_{i+1} = x_i +hx_i^{(1)}+\frac{h^2}{2}x_i^{(2)}+O(h^3)
\end{equation}

where $v_i^{(1)}$ is still the acceleration defined by the acting forces.
We can remove second derivative by substituting it with the first
derivative multiplied with the time step. 

\begin{equation}
    hv_i^{(2)} \approx v_{i+1}^{(1)}-v_i^{(1)}\nonumber
\end{equation}
This applied to our Taylor series gives us
\begin{align}
    v_{i+1} &= v_i + hv_i^{(1)}+\frac{h}{2}\left(v_{i+1}^{(1)}-v_i^{(1)}\right)\nonumber\\
    &= v_i+\frac{h}{2}\left(v_{i+1}^{(1)}+v_i^{(1)}\right)
\end{align}

As for our position calculation, we will use the definition of motion
derivatives, namely $v_i^{(1)}=x_i$.

\begin{align}
    x_{i+1} &= x_i +hx_i^{(1)}+\frac{h^2}{2}x_i^{2}+O(h^3)\nonumber\\
    &= x_i + hv_i + \frac{h^2}{2}v_i^{(1)}
\end{align}

As shown in figure \cref{fig:XXX} this method makes for a more precise
calculation of the trajectories. But with 9 FLOPS per time step, this
method is way more demanding in terms of computational calculation than the
aforementioned.

% \begin{figure}[htpb]
%     \centering
%     % \includegraphics[width=0.8\linewidth]{XXX.pdf}
%     \caption{XXX}
%     \label{fig:XXX}
% \end{figure}

\subsection{Relativistic corrections}
General relativity predicts corrections to the Newtonian force an object
experiences in the vicinity of a massive object. For Mercury, the
gravitational force is better approximated with the expression

\begin{equation}
    F_G = \frac{GM_{\Sun}M_{\Mercury}}{r^2}
    \left[1+\frac{3l^2}{r^2 c^2}\right] ,
\end{equation}

where  $M_\Sun$ is the mass of the Sun, $M_\Mercury$ is the
mass of Mercury, $l$ is the angular momentum of mercury around the sun
center of mass and $c$ is the speed of light in vacuum.

\subsection{Placeholder}


%%%%%%%%%%%%%%%%%%%%%%%%%%%%%%%%%%%%%%%%%%%%%%%%%%%%%%%%%%%%%%%%%%%%%%%%%%%
\section{Methods}

\subsection{An object oriented approach}
We implement three different classes in
\texttt{c++}; \texttt{CelestialBodies} represents the individual bodies as
point particles, storing their physical variables. This is again stored in
\texttt{SolarSystem}, which represents the environment and the physical
laws. Finally, \texttt{Integrator} is responsible for propagating the
system forward in time, using the forces and an numerical integration
method to find the object positions in the next time step.

The basis of this structure was laid by Anders Hafreager
\href{https://github.com/andeplane/solar-system}{here}, providing a basic
forward euler and declaration of many of the member variables and functions
needed for the project. 


\subsection{Two-body problem \texorpdfstring{$\Sun\Earth$}{}} 
We assume a circular orbit of the earth around the sun, and also that the
acceleration of the sun due to the earth is negligible. The
Sun-Earth-system is then initialized by setting the Sun to a fixed position
at $\vec r_\Sun = 0$, and the position of Earth to $\vec r_\Earth =
\SI{1}{AU}\hat e_x$  with velocity $\vec v_\Earth = \SI{2\pi}{AU/yr}\hat e_y$.

The solution of the orbit of Earth is found using both Forward Euler and
Velocity Verlet. 

\subsubsection{Stability analysis}
For only internal conservative forces, the total of kinetic and potential
energy of a physical system is conserved.  To test the integration algorithms, 
we calculate the energies using equations \cref{eq:energies}.

\subsection{Three-body problem \texorpdfstring{$\Sun\Earth\Jupiter$}{}} 
\subsubsection{Fixed Sun }
We analyse the stability of the solar system when including Jupiter
($\Jupiter$) in our previous setup. The initial position of Jupiter is set to
$\vec r_\Jupiter = -\SI{5.2}{AU}\hat e_x$, with velocity $\vec v_\Jupiter =
-\SI{1}{AU/yr}\hat e_y$, found from inserting numbers in \cref{eq:circVel}.
The mass of Earth is now relevant, as we will be calculating the
gravitational effects from all bodies on all the others except on the Sun.
The masses of the objects are listed in the appendix.

\subsubsection{Massive Jupiter }
We then increase the mass of Jupiter to study the effects on the stability
of earth. The mass is increased to $10x$ and $1000x$ the original mass of
Jupiter.
\subsubsection{Moving Sun }
\subsection{N-body problem \texorpdfstring{$\Sun\Mercury\Pluto\Earth\Mars\Jupiter\Saturn\Neptune\Uranus\Pluto$}{}} 
\subsection{Mercury perihelion precession \texorpdfstring{$\Sun\Mercury$}{}} 
\subsection{Testing and algorithm analysis}
\subsubsection{Stability of \texorpdfstring{$\Delta t$}{}}
\subsubsection{Energy and angular momentum conservation}
\subsubsection{FLOPS}

%%%%%%%%%%%%%%%%%%%%%%%%%%%%%%%%%%%%%%%%%%%%%%%%%%%%%%%%%%%%%%%%%%%%%%%%%%%
\section{Results}

%%%%%%%%%%%%%%%%%%%%%%%%%%%%%%%%%%%%%%%%%%%%%%%%%%%%%%%%%%%%%%%%%%%%%%%%%%%
\section{Discussion and conclusions}
\section*{Appendix}
\begin{table}[H]
    \caption{Masses of the bodies in the solar system}
    \centering
    \begin{tabular}{| c | c | c | }
        \hline
        Body & Mass $\SI{10e24}{kg}$ & Distance to sun in AU\\
        \hline
        Sun      $\Sun$   & \SI{1.989e6}{}&  0 \\
        Mercury  $\Mercury$ & 0.3301    &  0.39 \\
        Venus    $\Venus$   & 4.867     &  0.72 \\
        Earth    $\Earth$   & 5.972     &  1 \\
        Moon     $\Moon$    & 0.073     &  1 \\
        Mars     $\Mars$    & 0.642     &  1.52 \\
        Jupiter  $\Jupiter$ & 1898      &  5.3 \\
        Saturn   $\Saturn$  & 568       &  9.54 \\
        Uranus   $\Uranus$  & 86.8      & 19.19 \\
        Neptune  $\Neptune$ & 102       & 30.06 \\
        Pluto    $\Pluto$   & 0.0146    & 39.53 \\
        \hline
    \end{tabular}
    \label{tab:CelestialMasses}
\end{table}


\bibliography{bib1}{}
\bibliographystyle{plain}

\end{multicols}


\end{document}
