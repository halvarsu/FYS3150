\documentclass[12pt]{article}

\usepackage{amsmath}
\usepackage{graphicx}
\usepackage{esvect}
\usepackage[margin=0.7in]{geometry}
\usepackage{float}
\usepackage{listings}
\usepackage[utf8]{inputenc}
\usepackage[norsk]{babel}
\usepackage[parfill]{parskip}  

\begin{document}
\title{Schrödingerlikningen
\\ Prosjekt 2
\\ FYS3150}
\author{Ragnar Bruvoll \and Halvard Sutterud}
%\author{Halvard Sutterud}
\date{September 2017}
\maketitle{\begin{center}\end{center}}
\pagenumbering{gobble}
\thispagestyle{empty}

\begin{abstract}
    Vi studerer den Scrödingerlikningen for ett og to elektroner i et
    harmonisk oscillator-potensiale, og ser
\end{abstract}
\newpage
\pagenumbering{arabic}


% \section*{Teori}

Ser på den radielle Schrôdringerligninen for ett elektron. Den ser slik ut:

\begin{equation*}
  -\frac{\hbar^2}{2 m} \left ( \frac{1}{r^2} \frac{d}{dr} r^2
  \frac{d}{dr} - \frac{l (l + 1)}{r^2} \right )R(r) 
     + V(r) R(r) = E R(r)
\end{equation*}
der  $V(r) = (1/2)kr^2$ er harmonisk oscillator potensial, $k=m\omega^2$,
$\omega$ er oscillatorfrekvens, E er energien til den harmoniske
oscillatoren og tar verdiene

\begin{equation*}
E_{nl}=  \hbar \omega \left(2n+l+\frac{3}{2}\right),
\end{equation*}
der n og l er kvantetall og tar verdiene $n=0,1,2,\dots$ and
$l=0,1,2,\dots$. Ettersom vi bruker sfæriske coordinater vil $r\in
[0,\infty)$.\\

For å forenkle uttrykket kan vi sette $R(r) = (1/r) u(r)$, som gir

\begin{equation*}
  -\frac{\hbar^2}{2 m} \frac{d^2}{dr^2} u(r) + \left ( V(r) + \frac{l (l + 1)}{r^2}\frac{\hbar^2}{2 m}\right ) u(r)  = E u(r)
\end{equation*}
der $u(0)=0$ og $u(\infty)=0$ ettersom sannsynlighetskurven flater drastisk
ut når r stiger. Ved å substituere $\rho = (1/\alpha) r$ kan vi gjøre
variablen dimensjonsløs.

\begin{equation*}
  -\frac{\hbar^2}{2 m \alpha^2} \frac{d^2}{d\rho^2} u(\rho)+ \left ( V(\rho) + \frac{l (l + 1)}{\rho^2}\frac{\hbar^2}{2 m\alpha^2} \right ) u(\rho)  = E u(\rho)
\end{equation*}
Dersom vi setter momentet $l=0$ og plugger inn potensialet og multipliserer
med $2m\alpha^2/\hbar^2$ får vi følgende ligning

\begin{equation*}
  -\frac{d^2}{d\rho^2} u(\rho) 
       + \frac{mk}{\hbar^2} \alpha^4\rho^2u(\rho)  = \frac{2m\alpha^2}{\hbar^2}E u(\rho) .
\end{equation*}
Dersom vi bestemmer at $\alpha  = \left(\frac{\hbar^2}{mk}\right)^{1/4}$ og
lager variabel $\lambda = \frac{2m\alpha^2}{\hbar^2}E$ ender vi opp med en
Schrödringer-ligning på formen
\begin{equation*}
  -\frac{d^2}{d\rho^2} u(\rho) + \rho^2u(\rho)  = \lambda u(\rho) .
\end{equation*}

Som forklart i forrige prosjekt, kan den andrederiverte skrives på formen
\begin{equation}
    u''=\frac{u(\rho+h) -2u(\rho) +u(\rho-h)}{h^2} +O(h^2),
    \label{eq:diffoperation}
\end{equation}
Grenseverdiene for $\rho$, $\rho_{\mathrm{min}}=0$ og
$\rho_{\mathrm{max}}$, bestemmes av hvor elektronet kan befinne seg, altså
et sted mellom 0 og uendelig langt borte. Maksverdien må riktignok
forenkles til en mer håndterlig verdi. Skrittlengden vi bruker blir den
totale lengden delt på antall skritt.

\begin{equation*}
  h=\frac{\rho_N-\rho_0 }{N}.
\end{equation*}
Schrödingerligningen ser da slik ut:
\begin{equation}
-\frac{u(\rho_i+h) -2u(\rho_i) +u(\rho_i-h)}{h^2}+\rho_i^2u(\rho_i)  = \lambda u(\rho_i),
\end{equation}
som kan skrives på en enklere måte der $u(p_i+h)=u_{i+1}$ osv:

\begin{equation}
-\frac{u_{i+1} -2u_i +u_{i-1} }{h^2}+V_iu_i  = \lambda u_i,
\end{equation}

der $V_i=\rho_i^2$. Som forklart i forrige prosjekt kan dette settet av
lineære ligninger uttrykkes ved å bruke matriseregning med en
diagonalvektor som en vektor på hver side av diagonalen. Det første
elementet i den diagonale matrisen blir

\begin{equation*}
   d_i=\frac{2}{h^2}+V_i,
\end{equation*}

mens det første elementet utenfor diagonalen blir

\begin{equation*}
   e_i=-\frac{1}{h^2}.
\end{equation*}

og dermed kan Schrödringerligningen skrives som

\begin{equation*}
d_iu_i+e_{i-1}u_{i-1}+e_{i+1}u_{i+1}  = \lambda u_i,
\end{equation*}

Som vi ser vil kun $V_i$ påvirke verdien til matriseelementene.
Ligningssettet kan nå skrives som en ligning for egenverdier.

\begin{equation}
    \begin{bmatrix}
        d_0    & e_0   & 0     & 0      & \dots & 0     & 0 \\
        e_1    & d_1   & e_1   & 0      & \dots & 0     &0 \\
        0      & e_2   & d_2   & e_2    & 0     & \dots & 0\\
        \dots  & \dots & \dots & \dots  & \dots & \dots & \dots\\
        \dots  & \dots & \dots & \dots  & \dots & \dots & \dots\\
        0      & \dots & \dots & \dots  & e_{N-1}     &d_{N-1} & e_{N-1}\\
        0      & \dots & \dots & \dots  & \dots & e_{N} & d_{N}
    \end{bmatrix}
    \begin{bmatrix} 
        u_{0} \\ u_{1} \\ \dots\\ \dots\\ \dots\\ \dots\\ u_{N} 
    \end{bmatrix}
    =\lambda \begin{bmatrix} 
        u_{0} \\ u_{1} \\ \dots\\ \dots\\ \dots\\ \dots\\ u_{N}
    \end{bmatrix}.  
    \label{eq:sematrix}
\end{equation}

\end{document}
