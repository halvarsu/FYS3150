\documentclass[10pt]{article}

\usepackage{amsmath}
\usepackage{graphicx}
\usepackage[margin=0.7in]{geometry}
\usepackage{float}
\usepackage{listings}
\usepackage[utf8]{inputenc}
\usepackage[parfill]{parskip}  
\usepackage{multicol}
\usepackage{siunitx}
\usepackage{cleveref}
\usepackage{cite}
\usepackage{caption}

\captionsetup{width=0.6\linewidth}

\newcommand{\rhomax}{\rho_{\text{max}}}
\newcommand{\relerr}{\epsilon_{\text{rel}}}
\newcommand{\bigO}[1]{\mathcal{O}(#1)}

\graphicspath{{../results/}}

\begin{document}
\title{The Solar System
\\ Project 3
\\ FYS3150}
\author{Ragnar Bruvoll \and Halvard Sutterud}
%\author{Halvard Sutterud}
\date{October 2017}
\maketitle{\begin{center}\end{center}}
\pagenumbering{gobble}
\thispagestyle{empty}

\begin{abstract}
    We study a model of the solar system as gravitationally interacting
    celestial bodies.  We implement a method to solve the differential
    equations of the mechanics of the solar system in c++, and compare the
    effectiveness of two methods for numerical differentiation, namely
    Forward Euler and Velocity Verlet. Some properties of the solar system
    is studied, including escape velocity for a planet in a two body
    system, a modified three body problem, and finally a sun-mercury system
    with 
    
    The Velocity Verlet method is found to be a bit 
    % cases of one and two electrons trapped in a harmonic
    % oscillator. Jacobi's method for finding the eigenvalues of a matrix is
    % implemented, and applied to a discrete approximation of the
    % Schr\"{o}dinger equation. The analytic energies of a single electron 
    % is reproduced within a relative error of $\SI{e-4}{}$. Then the method
    % is applied to two interacting electrons in the same potential, where
    % analytic results from \cite{PhysRevA.48.3561} for certain
    % frequencies is reproduced.
\end{abstract}

\tableofcontents

\newpage
\pagenumbering{arabic}




\begin{multicols}{2}

\section{Theory of everything in our solar system}

\section{Introduction}
The purpose of this project is to solve the differential equations of the
mechanics of the solar system. We do this by implementing classes in 

    The purpose of this project is to develop a general algorithm that
    solves a set of linear equations by using Jacobi's method of finding
    eigenvalues. 

We're starting with a tridiagonal matrix from which we seek to find the
eigenvalues and eigenvectors. The method we will use "rotates" the the
matrix by a specific angle $\theta$ such that it gets closer to a diagnoal
matrix by making all the off-diagonal elements smaller each step. In the
end we're left with a diagonal matrix and and the corresponding eigenvalues
and eigenvectors.

We will use the program to solve a set of linear equations derived from
Schrödringer's equation for two electrons stuck in a three-dimentional
harmonic oscillator with and without Coloumb interaction. In order to do
this in an effective fashion, we will show how to scale and discretize the
original equation to a more manageable one.



%%%%%%%%%%%%%%%%%%%%%%%%%%%%%%%%%%%%%%%%%%%%%%%%%%%%%%%%%%%%%%%%%%%%%%%%%%%
\section{Theory/Methods}
\subsection{Units}
 % \subsection{Discretizing equations and defining acceleration}
Using Newton's second law, which states that the sum of all the forces
equals mass times acceleration, $\sum F=ma$, we can calculate the
acceleration of the planet in relation to the sun. Assuming circular
orbits, we know this acceleration to be the sentripetal acceleration, and
the force is the gravitational force

\begin{equation}
    F_G = M_Ea = M_E\frac{v^2}{r} = G\frac{M_OM_E}{r^2}
\end{equation}



with the earth as an example. With this in mind, we can simplify the
equations and calculations by using astronomical units and years as unit
lengths. So in stead of using $r = 1.5\cdot10^{11}m$ and , we'll simply
assert $r = 1AU$. With simple multiplication, the expressions for
sentripetal force and gravitational force can be written as 

\begin{equation}
    \Rightarrow v^2r = GM_O
\end{equation}

and with the new unit lengths applied we can simplify it further. We know
that the radius from the earth to it's spin system's centre is one
astronomical unit, 1 AU. The orbit speed is one round per year, that is
2$\pi$AU/year. This means

\begin{equation}
    GM_O = v^2r = 4\pi^2AU^3/year^2
\end{equation}

We can now subtitute this constant and get an expression for the general
acceleration of any object with an circular orbit around the sun

\begin{equation}
    M_Ea = GM_O\frac{M_E}{r^2}
\end{equation}

\begin{equation}
    a = \frac{4\pi^2}{r^2}
\end{equation}
with the acceleraion as expression of astronomical units per year squared. 

\subsection{Perihelion precession}
\subsection{Placeholder}

%%%%%%%%%%%%%%%%%%%%%%%%%%%%%%%%%%%%%%%%%%%%%%%%%%%%%%%%%%%%%%%%%%%%%%%%%%%
\section{Methods}

\subsection{Two-body problem} 
\subsubsection{Stability analysis}
\subsection{Three-body problem} 
\subsubsection{Fixed sun}
\subsubsection{Massive Jupiter}
\subsubsection{Moving sun}
\subsection{N-body problem - Full solar system model} 
\subsection{Mercury perihelion precession}
\subsection{Testing and algorithm analysis}
\subsubsection{Stability of $\Delta t$}
\subsubsection{Energy and angular momentum conservation}
\subsubsection{FLOPS}

%%%%%%%%%%%%%%%%%%%%%%%%%%%%%%%%%%%%%%%%%%%%%%%%%%%%%%%%%%%%%%%%%%%%%%%%%%%
\section{Results}

%%%%%%%%%%%%%%%%%%%%%%%%%%%%%%%%%%%%%%%%%%%%%%%%%%%%%%%%%%%%%%%%%%%%%%%%%%%
\section{Discussion and conclusions}
\bibliography{bib1}{}
\bibliographystyle{plain}

\end{multicols}


\end{document}
