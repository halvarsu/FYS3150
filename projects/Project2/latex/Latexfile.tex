\documentclass[10pt]{article}

\usepackage{amsmath}
\usepackage{graphicx}
\usepackage[margin=0.7in]{geometry}
\usepackage{float}
\usepackage{listings}
\usepackage[utf8]{inputenc}
\usepackage[parfill]{parskip}  
\usepackage{multicol}
\usepackage{siunitx}
\usepackage{cleveref}
\usepackage{cite}

\newcommand{\rhomax}{\rho_{\text{max}}}
\graphicspath{{../results/}}

\begin{document}
\title{The Schrödinger equation
\\ Project 2
\\ FYS3150}
\author{Ragnar Bruvoll \and Halvard Sutterud}
%\author{Halvard Sutterud}
\date{September 2017}
\maketitle{\begin{center}\end{center}}
\pagenumbering{gobble}
\thispagestyle{empty}

\begin{abstract}
    We study the cases of one and two electrons trapped in a harmonic
    oscillator. Jacobi's method for finding the eigenvalues of a matrix is
    implemented, and applied to a discrete approximation of the
    Schr\"{o}dinger equation. The analytic energies of a single electron 
    is reproduced within a relative error of $\SI{e-4}{}$. Then the method
    is applied to two interacting electrons in the same potential, where
    analytic results from \cite{PhysRevA.48.3561} for certain
    frequencies is reproduced.

\end{abstract}

\tableofcontents

\newpage
\pagenumbering{arabic}




\begin{multicols}{2}


\section{Introduction}
    The purpose of this project is to develop a general algorithm that
    solves a set of linear equations by using Jacobi's method of finding
    eigenvalues. 

We're starting with a tridiagonal matrix from which we seek to find the
eigenvalues and eigenvectors. The method we will use "rotates" the the
matrix by a specific angle $\theta$ such that it gets closer to a diagnoal
matrix by making all the off-diagonal elements smaller each step. In the
end we're left with a diagonal matrix and and the corresponding eigenvalues
and eigenvectors.

We will use the program to solved a set of linear equations derived from
Schrödringer's equation for two electrons stuck in a three-dimentional
harmonic oscillator with and without Coloumb interaction. In order to do
this in an effective fashion, we will show how to scale and discretize the
original equation to a more manageable one.



\section{Theory/Methods}
\subsection{Making the matrix}
The analytical Schroedringer equation is as follows
\begin{equation*}
  -\frac{\hbar^2}{2 m} \left ( \frac{1}{r^2} \frac{d}{dr} r^2
  \frac{d}{dr} - \frac{l (l + 1)}{r^2} \right )R(r) 
     + V(r) R(r) = E R(r)
\end{equation*}
where  $V(r) = (1/2)kr^2$ is the harmonic oscillator potential, k is the
displacement factor where $k=m\omega^2$.  $\omega$ is the frequence of the
oscillator and m is the mass of whatever's oscillating, in this case the
electron. E is the energy og the oscillator and will take the values

\begin{equation*}
E_{nl}=  \hbar \omega \left(2n+l+\frac{3}{2}\right),
\end{equation*}

due to quantum mechanical effects, where n and l are quantum numbers with
the following possible values $n=0,1,2,\dots$ and $l=0,1,2,\dots$. R is the
wave function, dependant on distance r. We'll be using spherical
coordinates, such that $r\in [0,\infty)$ is the distance from the center of
the oscillator to the position of the electron.\\

In order to simplify the equation, we'll make the new variable $u(r)$ such
that $R(r) = (1/r) u(r)$. This gives

\begin{equation*}
  -\frac{\hbar^2}{2 m} \frac{d^2}{dr^2} u(r) + \left ( V(r) + \frac{l (l + 1)}{r^2}\frac{\hbar^2}{2 m}\right ) u(r)  = E u(r)
\end{equation*}

the next step in generalizing the equation will be making the variable
dimentionless by introducing a new variable $\rho = (1/\alpha) r$ where
$u(0)=0$ and $u(\infty)=0$. The equation is now a function of the
dimentionless $\rho$.

\begin{equation*}
  -\frac{\hbar^2}{2 m \alpha^2} \frac{d^2}{d\rho^2} u(\rho)+ \left ( V(\rho) + \frac{l (l + 1)}{\rho^2}\frac{\hbar^2}{2 m\alpha^2} \right ) u(\rho)  = E u(\rho)
\end{equation*}
We will be looking at a case with no angular momentum, $l=0$. The last step
will be inserting the potential and removing all constants from the second
derivative by multiplying with $2m\alpha^2/\hbar^2$. This leaves us with

\begin{equation*}
  -\frac{d^2}{d\rho^2} u(\rho) 
       + \frac{mk}{\hbar^2} \alpha^4\rho^2u(\rho)  = \frac{2m\alpha^2}{\hbar^2}E u(\rho) .
\end{equation*}

Since $\alpha$ is an introduced constant, we can set it to $\alpha  =
\left(\frac{\hbar^2}{mk}\right)^{1/4}$ and define a new eigenvalue $\lambda
= \frac{2m\alpha^2}{\hbar^2}E$. This will result in a new Schrödringer
equation which is a lot easier to handle numerically;

\begin{equation*}
  -\frac{d^2}{d\rho^2} u(\rho) + \rho^2u(\rho)  = \lambda u(\rho) .
\end{equation*}
As explained in the previous project (Project 1), the second derivative can
be expressed as


\begin{equation}
    u''=\frac{u(\rho+h) -2u(\rho) +u(\rho-h)}{h^2} +O(h^2),
    \label{eq:diffoperation}
\end{equation}
where the interval of $\rho$ is $\rho_{\mathrm{min}}=0$ and
$\rho_{\mathrm{max}}$ correlates to the point where the wave function is
reasonably close to zero. Theoretically it should be set to infinity, but
this is practically difficult to enforce. The step length of the numerical
second derivative function will then be the total length of the distance
we're working devided by number of steps.

\begin{equation*}
  h=\frac{\rho_N-\rho_0 }{N}.
\end{equation*}
This gives us a set of discrete values for the variable $\rho_i = ih$ since
$\rho_0 = 0$\\\\ By now, put equation reads

\begin{equation}
-\frac{u(\rho_i+h) -2u(\rho_i) +u(\rho_i-h)}{h^2}+\rho_i^2u(\rho_i)  = \lambda u(\rho_i),
\end{equation}
which can by compacted by doing the following transformation
$u(p_i+h)=u_{i+1}$ etc.:

\begin{equation}
-\frac{u_{i+1} -2u_i +u_{i-1} }{h^2}+\rho_i^2u_i  = \lambda u_i,
\end{equation}

This will result in a set of linear equations which can be expressed
through matrix calculations. It will be an NxN triagonal matrix with values
only on the diagonal and directly underneath and above the diagnoal. The
rest will be zero. The diagnoal elements will be

\begin{equation*}
   d_i=\frac{2}{h^2}+V_i,
\end{equation*}

and the rest of the non-zero will be

\begin{equation*}
   e_i=-\frac{1}{h^2}.
\end{equation*}

Finally, we're left with an eigenvalue equation


\begin{equation*}
    \begin{bmatrix}
        d_0    & e_0   & 0      & \dots & 0     & 0 \\
        e_1    & d_1   & e_1    & \dots & 0     &0 \\
        0      & e_2   & d_2    & 0     & \dots & 0\\
        \dots  & \dots & \dots  & \dots & \dots & \dots\\
        0      & \dots & \dots  & e_{N-1}     &d_{N-1} & e_{N-1}\\
        0      & \dots & \dots  & \dots & e_{N} & d_{N}
    \end{bmatrix}
    \begin{bmatrix} 
        u_{0} \\ u_{1} \\ \dots\\ \dots\\ \dots\\ u_{N} 
    \end{bmatrix}
    =\lambda \begin{bmatrix} 
        u_{0} \\ u_{1} \\ \dots\\ \dots\\ \dots\\ u_{N}
    \end{bmatrix}.  
    \label{eq:sematrix}
\end{equation*}
and our Schroedringer equation is
\begin{equation*}
d_iu_i+e_{i-1}u_{i-1}+e_{i+1}u_{i+1}  = \lambda u_i,
\end{equation*}
where $u$ is unknown. 
\subsection{Jacobi rotation algorithm}
In order to find the eigenvalues of this matrix, we will use Jacobi's
method, which is based on turning our matrix into a tri diagonal one with
unitary transformations. We can do this with the following equation
\begin{align*}
    \textbf{B} = \textbf{S}^T\textbf{AS}
\end{align*}
where $\textbf{A}$ is our original matrix and $\textbf{B}$ is the
tridiagonal matrix we wish to find. $\textbf{S}$ and
$\textbf{S}^T=\textbf{S}^{-1}$ are our transformation matrices. The process
will rotate the 2-dimentional linear subspace such that all the
non-diagonal elements will get closer to zero for each rotation.  The
$\textbf{S}$ matrix has has the form
\begin{align*}
    \textbf{S} &=
    \begin{bmatrix}
        1     & 0     & 0     & \dots  & 0 & 0 & \dots & 0 & 0\\
        0     & 1     & 0     & \dots  & 0 & 0 & \dots & 0 & 0\\
        0     & 0     & 1     & \dots  & 0 & 0 & \dots & 0 & 0\\
        \dots & \dots & \dots & \dots & \dots & \dots & \dots& \dots & \dots\\
        0 & 0 & 0 & \dots & cos\theta & 0 & \dots& 0 & sin\theta\\
        0     & 0 & 0 & \dots & 0 & 1 &\dots & 0 & 0\\
        \dots & \dots & \dots & \dots & \dots & \dots & \dots& \dots & \dots\\
        0     & 0 & 0 & \dots  & 0 & 0 & \dots & 1 & 0\\
        0     & 0 & 0 & \dots  & -sin\theta & 0 & \dots &  0 & cos\theta
    \end{bmatrix}
\end {align*}
and does the rotation around the angle $\theta$ which is chosen such that
$cot(2\theta) = \frac{a_{ll}-a_{kk}}{2a_{kl}}$. The matrix element $a_{kl}$
will be the largest element outside the diagonal and the sinus og cosinus
is determined by  


\begin{align*}
    \tau &= cot(2\theta) = \frac{a_{ll}-a_{kk}}{2a_{kl}}\\
    t^2 &+2\tau t -1 = 0\\
    \rightarrow t &= -\tau\pm \sqrt{1+\tau^2}\\
    c &= \frac{1}{\sqrt{1+t^2}}\\
    s &= tc
\end{align*}.
Once the angles are known, the corresponding matrix elements will be given
the new values
\begin{align*}
    b_{ik} &= a_{ik}cos\theta-a_{il}sin\theta\ \ i\neq k, i \neq l\\
    b_{il} &= a_{il}cos\theta-a_{ik}sin\theta\ \ i\neq k, i \neq l\\
    b_{kk} &= a_{kk}cos^2\theta-2a_{kl}cos\theta sin\theta + a_ {ll}sin^2\theta\\
    b_{ll} &= a_{ll}cos^2\theta+2a_{kl}cos\theta sin\theta + a_ {kk}sin^2\theta\\
    b_ {kl} &= (a_{kk}-a_{ll})cos\theta sin\theta +a_{kl}(cos^2\theta-sin^2\theta)
\end{align*}
By doing this process enough times, the non diagonal elements will be close
or equal to zero, and we're left with a diagnoal matrix containing the
eigenvalues of our wave function.\\

A way to ensure that our matrix converges to a diagonal matrix and is
preserving its properties, we can make sure that the Frobenius norm is
preseved. The Frobenius norm is defined
\begin{align*}
    \left|\left|\textbf{A}\right|\right|_F &= \sqrt{\sum_{i=1}^n\sum_{j=1}^n|a_{ij}|^2}
\end{align*}
and the aim is to reduce the norm of the off-diagonal matrix elements thus
getting closer to a diagonal matrix step by step. Using a 2x2 matrix for
simplicity, our transformation will take the form
\begin{align*}
    \begin{bmatrix}
        b_{kk} & 0\\
        0 & b_{ll}
    \end{bmatrix}&=
    \begin{bmatrix}
        c & -s\\
        s & c
    \end{bmatrix}
    \begin{bmatrix}
        a_{kk} & a_{kl}\\
        a_{lk} & a_{ll}
    \end{bmatrix}
    \begin{bmatrix}
        c & s\\
        -s & c
    \end{bmatrix}\\
\end{align*}
Meaning
\begin{align*}
    b_{kk}^2+b_{ll}^2 &= a_{kk}^2+a_{ll}^2+2a_{kl}^2
\end{align*}
since B is the diagonal matrix and A is symmetrical.
\begin{align*}
    &\boxed{||\textbf{B}||_F^2 = off(\textbf{B})^2+\sum_{i=1}^nb_{ii}^2 =||\textbf{A}||_F^2 = off(\textbf{A})^2+\sum_{i=1}^na_{ii}^2}\\
\end{align*}
\begin{align*}
    off(\textbf{B})^2 &= ||\textbf{B}||_F^2 -\sum_{i=1}^nb_{ii}^2
    = off(\textbf{A})^2 + \sum_{i=1}^na_{ii}^2 -\sum_{i=1}^nb_{ii}^2\\
    &= off(\textbf{A})^2+\left(a_{kk}^2+a_{ll}^2+b_{kk}^2+b_{ll}^2\right)\\
    ||\textbf{B}||^2-\sum_{i=1}^nb_{ii} &= norm(\textbf{A})_F^2-\sum_{i=1}^na_{ii}^2\left(a_{kk}^2+a_{ll}^2+b_{kk}^2+b_{ll}^2\right)
\end{align*}
which will give us an $\textbf{A}$ which is closer to a diagonal matrix
than before the transformation.



\section{Methods}
\subsection{Testing}
In order to detect, repair and prevent errors in algorithms, implementing
unit tests is common practice. The earlier these tests are implemented, the
less time is wasted on locating the source of the errors. Even small errors
can accumulate to big ones with enough iterations. For this project there
are several simple tests that can confirm the validity of the program.  

\subsubsection{Finding the right element}
To ensure that $\textbf{B}$ is converging towards a diagonal matrix, and to
make the algorithm as effective as possible, it is important that we locate
the largest off-diagonal matrix element for every rotation. By setting any
element $a_{ij}\neq a_{ii}$ to a number known to be the largest in that
matrix, we can test if the algorithm finds this number by printing k and l.
If these correspond to i and j, we know they're correct.

\subsubsection{Setting right element to zero}
Now that we're sure the code finds the right indices, it's easy to test if
the right elements are being nulled. We do this by setting $a_{kl}$ to a
non-zero number, running the Jacobi rotate function and printing the same
element. If the print is anything else than zero, we know something is
wrong.

\subsubsection{Testing if the eigenvalues}
By creating a simple 3x3 matrix with known eigenvalues and running it
through the algorithm, we can test if the result checks out. Although this
won't consider the numerical errors a big matrix and a large number of
transformations would accompany, the error will be detected by our
tolerance, set to $10^{-10}$. The test funciton also includes a test of the
Frobenius norm, which checks that $||A_{before}||^2 = ||A_{after}||^2$, the
Frobenius norm is preserved.

\subsubsection{Testing orthonormality of eigenvectors}
Since what we're really looking for are the eigenvectors, we would be
remiss not to test their orthonormality, as this is a criteria for any
diagonal matrix. This can easily be done by multiplying the matrix with its
transposed and checking if this is 1 or close enough. If $\textbf{B}$ is
orthonormal, it should give the identity matrix if multiplied with its
transposed. By testing if $\textbf{B}\cdot\textbf{B}^T -\textbf{I}<
10^{-10}$ holds (for all elements of B), we know that $\textbf{B}$ is
orthonormal. The tolerance is there to accept negligible numerical errors.


\input{methods.tex}
\section{Results}
\subsection{Stability analysis}
\label{sub:a_}

The stability analysis of the jacobi solver did not converge to the
analytical value for increasing $\rhomax$, as this increased the step size
and caused reduced precision. For $N=200$, a $\rhomax$ of $\approx 5$ was
found ideal for the third wavefunction in the non interacting harmonic
oscillator. As seen in \cref{fig:rhoMax1} and \ref{fig:rhoMax2}, the
critical value for $\rhomax$ is  Then setting $\rhomax$ equal to 



\begin{figure}[H]
    \centering
    \includegraphics[width=0.8\linewidth]{iterations.pdf}
    \caption{Iterations of jacobi rotate vs N}
    \label{fig:iterations}
\end{figure}

\begin{figure}[H]
    \centering
    \includegraphics[width=0.8\linewidth]{rhoMaxAnalysis1.pdf}
    \caption{Convergence of $\rhomax$. As}
    \label{fig:rhoMax1}
\end{figure}

\begin{figure}[H]
    \centering
    \includegraphics[width=0.8\linewidth]{rhoMaxAnalysis2.pdf}
    \caption{Convergence of $\rhomax$ }
    \label{fig:rhoMax2}
\end{figure}

\begin{figure}[H]
    \centering
    \includegraphics[width=0.8\linewidth]{dimAnalysis1.pdf}
    \caption{The normalized wavefunction in the testing the values of $N$.
    All values give visually similar wavefunctions.}
    \label{fig:dim1}
\end{figure}

\begin{figure}[H]
    \centering
    \includegraphics[width=0.8\linewidth]{dimAnalysis2.pdf}
    \caption{The normalized wavefunction in the testing the values of $N$.
    All values give visually similar wavefunctions.}
    \label{fig:dim2}
\end{figure}

\begin{figure}[H]
    \centering
    \includegraphics[width=0.8\linewidth]{waveFunc.pdf}
    \caption{Normalized wave function of the relative coordinate between
        the electron for the interacting case with $\omega_r \in\{0.25,
    0.5, 1.0, 5.0\}$.}
    \label{fig:waveFunc}
\end{figure}




\section{Conclusions}
\input{conclusions.tex}
% \section{References}

\bibliography{bib1}{}
\bibliographystyle{plain}

\end{multicols}


\end{document}
